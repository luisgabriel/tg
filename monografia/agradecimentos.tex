Primeiramente, para ser justo, preciso agradescer à minha família. Aos meus pais, Algeny e Alcides, muito obrigado pelas ótimas condições que vocês me proporcionaram durante toda minha vida. Muito obrigado pela paciência, pelo amor e pelo carinho que vocês tem comigo. Agradeço também à minha irmã, Isabela, pelo carinho e pelo zelo que você tem por mim. Sem o apoio de vocês eu não teria chegado até aqui. Muito obrigado também pela compreensão durante os vários finais de semana que estive ausente devido à compromissos da faculdade.

Gostaria de agradecer aos meus amigos Cláudio (Rabicó), Gustavo, Lucas (Tibuta), Marcela, Marina (Nina) e Tiago (Rato) pelas conversas, pelas viagens, pelas risadas e pelos vários momentos que compartilhamos juntos. Vocês são pessoas fantásticas! De formas diferentes cada um de vocês contribiu muito para formação do meu intelécto e do meu caráter. Agradeço também a meu amigo Rafael (Bobinho) pelas ótimas conversas regadas ao bom e velho blues na Praça de Casa Forte.

Agradeço também à Marcelo Jensen por sempre consertar as besteiras que eu fazia no computador de casa quando eu era mais novo. Sua paciência em sempre me explicar o que estava acontecendo e o porque das coisas não estarem funcionando como deveriam foram fundamentais para despertar em mim o interesse pela computação.

Agradeço à Bruno Coelho, Crystal, Lívia e Tiago por terem sido mais que amigos durante esse período da graduação. Agradeço à Bruno Medeiros, Henrique, Helder e Jonathas pela amizade e pela companhia durante os vários projetos que fizemos juntos. Agradeço também aos demais colegas de turma de CC e EC pela convivência durante esse período, sem todos vocês teria sido muito mais difícil!

Agradeço ao meu orientador, Fernando Castor pelo apoio e pela inspiração profissional durante esses quase três anos nos quais fui seu aluno, monitor e orientando. Agradeço ao professor Ricardo Massa pela confiança depositada em mim durante a monitoria de Introdução à Programação. Agradeço ao professor Paulo Gonçalves pelos ensinamentos valiosos durante o período de iniciação científica. Agradeço também a todo corpo docente do CIn pelo empenho e dedicação que foram fundamentais para minha formação acadêmica e pessoal.

Por fim, agradeço ao Instituto Nokia de Tecnologia (INdT) por ser um lugar fantástico de se trabalhar. Muito obrigado às várias pessoas com que tive o prazer de trabalhar durante esses dois anos de INdT e que hoje servem de inspiração pessoal e profissional para mim. Em especial, gostaria de agradecer à Anselmo, Daker, Lacerda, Figueredo, Mailson, Adriano e Cidorvan pelas várias discussões técnicas e pelas conversas aleatórias durante o almoço.

Os processadores \emph{multicore} estão presentes em praticamente todos os computadores modernos, inclusive em dispositivos móveis como telefones celular e \emph{tablets}. Porém, pouco desse poder de processamento, provido pelos múltiplos núcleos, é aproveitado de maneira efetiva pelas aplicações devido à dificuldade de se escrever sistemas concorrentes. Com o objetivo de tornar o desenvolvimento desse tipo de sistema mais palpável, alguns novos mecanismos de sincronização e paralelismo vem sendo propostos em linguagens de programação funcional. Esse tipo de linguagem prega um estilo de programação baseado em funções puras e dados imutáveis que facilita o desenvolvimento de programas concorrentes. Com o objetivo de entender melhor esses benefícios, este trabalho faz um estudo comparativo entre as linguagens funcionais Clojure e Haskell com foco na utilização de Memória Transacional em Software. Para isso foi utilizado como objeto de estudo a implementação de um motor de busca paralelo em ambas as linguagens.

\begin{keywords}
memória transacional em sotware, programação funcional, concorrência, motor de busca, recuperação de informação
\end{keywords}
